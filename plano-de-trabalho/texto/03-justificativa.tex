
\subsection{Justificativa}


O Ministério da Cultura vem nos últimos 2 anos executando uma Plano Diretor de TI - PDTI que busca implementar ações inovadoras através da
realização de projetos em parceria com universidades. No último ano vimos executando parcerias bem sucedidas com a UFABC, UFG, UFPR e UFPB.
O referido Plano tem como um dos objetivos estratégicos o seguinte: “9. Prospectar junto com universidades e centros de P\&D novas formas de
desenvolvimento de software”. Este objetivo estratégico preconiza que o modelo de desenvolvimento de software praticado, conhecido como fábrica
de software, é incapaz de oferecer soluções que atendam às necessidades reais dos gestores públicos e demais usuários. A realização de parceria
com universidades busca aliar o potencial inovador da academia com as necessidades reais da gestão pública através de um modelo de desenvolvimento
laboratorial que tem apresentado resultados significativos.

Temos conhecimento de que o Laboratório Avançado de Pesquisa e Desenvolvimento de Software (LAPPIS) da Faculdade UnB Gama (FGA) estabeleceu um
método que facilita a participação de alunos do curso de engenharia de software em projetos cívicos de software livre como instrumento pedagógico. 
Coordenado por professores com experiência na participação de comunidades, desenvolvimento de softwares livres e métodos ágeis, foi o parceiro de 
desenvolvimento de projetos importantes do governo federal como o Portal do Software Público, Participa.br, Aplicativo da Conferência da Juventude e
Dialoga Brasil.

As especialidades adquiridas pelo LAPPIS no âmbito da engenharia de software aliadas às praticas metodológicas da cultura do software livre, em
especial a experiência no desenvolvimento da plataforma de gestão da política de software livre do governo, denominado Portal Software Público 
Brasileiro, demonstram as capacidades necessárias para a realização de uma ação multidisciplinar que visa modernizar o processo de desenvolvimento
de software no MinC, além de aprimorar algumas das principais plataformas tecnológicas através do uso de tecnologias de ponta para análise de dados 
e aprendizagem de maquina.

Dessa forma, considerando a necessidade do Ministério de estabelecer parcerias para o desenvolvimento deste projeto de participação social, a
colaboração com a UnB, no sentido de estabelecer uma parceria de inovação e pesquisa visando estudo e desenvolvimento de tecnologias livres para 
ambiente digital e de colaboração.
