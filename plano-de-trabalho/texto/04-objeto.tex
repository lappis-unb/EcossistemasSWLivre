\section{Objeto}
\label{objeto}


%produto do convênio, contrato de repasse ou termo de cooperação, observados o programa de trabalho e as suas finalidades;
% 
% Uso de softwares livres para apoiar a gestão governamental e
% das políticas públicas do Ministério da Cultura. Inovação suportada por metodologias ágeis e plataformas abertas e colaborativas:
% pesquisa de novos modelos de desenvolvimento de software para o Ministério da Cultura e suas vinculadas.

Constituir a rede de laboratórios de pesquisa e desenvolvimento de tecnologias inovadoras para políticas públicas do Ministério da Cultura.
O principal objetivo é pesquisar e aplicar técnicas, metodologias de desenvolvimento de software, além de aferição qualidade produto de software, em ambiente experimental do Laboratório Avançade de Pesquisa, Produção e Inovação em Software (LAPPIS). 
Tais pesquisas e práticas serão usadas para subsidiar o Ministério da Cultura de 
ferramentas de gestão e desenvolvimento de software colaborativo, aberto e contínuo, em diferentes arranjos produtivos, 
aprimorando os mecanismos de governança digital; além de fornecer subsídios tecnológicos que apoiem a execução da lei Lei 8.313/91,
conhecida como Rouanet e das demais políticas de fomento e incentivo à cultura.  


% \subsection{Descrição}
% 
% % descrição detalhada do objeto, indicando os programas por ele abrangidos
% 
% Atualmente, algumas organizações da Administração Pública Federal iniciam investimentos para adotar contratações de serviços de
% desenvolvimento de software utilizando métodos ágeis e desenvolvimento colaborativo, motivadas pelo entendimento
% de que os instrumentos contratuais, hoje em vigor, apresentam limitações que causam impacto nos custos dos projetos, 
% na entrega do produto de software e satisfação do usuário final.
% 
% A partir desse contexto, no decorrer da execução deste projeto, será construída
% uma rede de organizações acadêmicas, públicas, sociedade civil e terceiro setor,
% com o objetivo principal de colaborar no desenho e suporte de um novo arcabouço legal e
% técnico para desenvolvimento de aplicações públicas e livres voltados para o
% Ministério da Cultura e suas vinculadas. 
% 
% Trata-se de um arranjo de pesquisa, de ação e inovação sobre a realidade de engenharia
% de software em organizações da Administração Pública Federal. Para responder a pergunta de pesquisa proposta,
% no contexto de projetos de softwares do Ministério da cultura, será  estado da prática em engenharia de software
% deve ser executado a fim de 
% 
% Seu grau de complexidade e a exigência de competências tão diversas coloca-nos
% o desafio da integração inovadora não somente entre diferentes campos científicos, mas também
% novos modos de articulação Estado-Universidade-Socidade para a execução e analise de políticas
% publicas no cenário de uma sociedade informacional
% 
% 
