\section{Introdução}
\label{introducao}

A parceria entre a Universidade de Brasília e o Ministério da Cultura visa aliar 
competências em desenvolvimento e software, domínio de tecnologias livres e 
métodos ágeis do Laboratório Avançado de Produção, Pesquisa e Inovação em Software (LAPPIS)
com a expertise do Ministério da Cultura em fornecer aos municipios e estados 
as funcionalidades de software que apoiem a execução da lei Lei 8.313/91,
conhecida como Rouanet e das demais políticas de fomento e incentivo à cultura. 

Trata-se de um arranjo interdisciplinar que envolve pesquisa e inovação em 
processos de desenvolvimento de soluções que serão incorporadas à esfera 
pública na forma de Softwares Livres e de modo mais concreto nos mecanismos de 
governança digital do próprio Ministério da Cultura. 

A Tecnologia da Informação e Comunicação (TIC) está adquirindo papel cada vez 
mais central no cotidiano das organizações sociais. De governos, grandes empresas, 
autarquias, passando por organizações de sociedade civil até mesmo por movimentos 
informais, as TICs tem sido o principal meio de comunicação e organização da 
ação coletiva nesses contextos. 

Por isso, o software é algo de interesse geral, uma vez que vários aspectos
relacionados a ele ultrapassam as questões técnicas, como por exemplo: o
processo de desenvolvimento do software; os mecanismos econômicos que regem
esse desenvolvimento e seu uso; o relacionamento entre desenvolvedores,
fornecedores e usuários do software; os aspectos éticos e legais relacionados
ao software~\cite{kon2011jai}.

Software livre é aquele que permite aos usuários 
usá-lo, estudá-lo, modificá-lo e redistribui-lo, sem restrições. Normalmente, 
esse software existe por meio de projetos de desenvolvimento que estão centradas 
em torno de algum código-fonte acessível ao público, geralmente em um repositório na Internet,
onde desenvolvedores e usuários podem interagir~\cite{meirelles2013}. O código
é necessariamente licenciado sob termos legais formais que estão de acordo com
as definições da Free Software Foundation
\footnote{\url{gnu.org/philosophy/free-sw.html}} ou da Open Source Initiative
\footnote{\url{opensource.org/docs/definition.html}}.


O que define e diferencia o software livre do que podemos denominar de software
restrito passa pelo entendimento desses quatro pontos dentro do que é conhecido
como o ecossistema do software livre~\cite{meirelles2013}. O princípio básico 
desse ecossistema é promover a liberdade do usuário, sem
discriminar quem tem permissão para usar um software e seus limites de uso,
baseado na colaboração e num processo de desenvolvimento
aberto~\cite{kon2011jai}. O desenvolvimento de software livre é uma alternativa estratégica
para o Estado para
\begin{itemize}
 \item Soberania;
 \item reuso e colaboração;
 \item Autonomia;
 \item Inteligência institucional.
\end{itemize}

O desenvolvimento de software livre é baseado num método que promove a transparência (pelas licenças) e a participação
(pelo método baseado no acesso aberto ao processo de desenvolvimento) desde o momento de concepção. Por conta disso entendemos 
que essa é a forma mais adequada em um projeto que visa promover a colaboração de outras organizações e da cidadania, 
de forma aberta e transparente, no desenvolvimento de tecnologia do ministério.

O desenvolvimento de software livre é uma alternativa estratégica para o Estado por contribuir para o reuso de tecnologias que já
tenham sido desenvolvidas e possam ser apropriadas e aprimoradas pelas instituições públicas, promovendo economia de recursos e acesso a
tecnologias de ponta com desenvolvimento ativo. As características do licenciamento de software livre e seu processo de documentação pública com formação de comunidade também melhora o nível de autonomia do Estado, já que pode se apropriar da tecnologia desenvolvida a partir do seu próprio corpo técnico ou mobilizando a riqueza da comunidade. Esse processo vai aprimorando e capacitando os recursos humanos dos orgãos públicos aumentando a inteligência institucional e contribuindo em questões de soberania nacional, no sentido de que as tecnologias adotadas para a gestão pública são baseadas em inteligências presentes no Estado e sociedade Brasileira.

É importante, portanto, implementar uma arquitetura que facilita a participação, seja pelos métodos, seja promovendo ações que motivem o engajamento de atores, visando alimentar um processo que vai construindo, gradualmente, um ambiente de colaboração na construção das tecnologias que suportam a gestão cultural no Brasil.

Nesse contexto, o Ministério da Cultura possui histórico no desenvolvimento e manutenção de sistemas
software livre para políticas públicas\footnote{\url{https://github.com/culturagovbr}}.
Dentre os 27 sistemas desenvolvidos e/ou mantidos, todos disponibilizados como software livre,
se destacam:

\begin{itemize}
\item \textbf{Salic (Sistema de Apoio às Leis de Incentivo à Cultura)} - 
 principal mecanismo de fomento à Cultura do Brasil, a Lei 8.313/91 (Lei Rouanet) 
 estabelece as normativas de como o Governo Federal deve disponibilizar recursos 
 para a realização de projetos artístico-culturais. A implementação da Lei 
 Rouanet, tanto a fase de habilitação quanto prestação de conta, é feita por 
 meio do sistema Salic.
 \item \textbf{Mapas Culturais}\footnote{\url{https://github.com/culturagovbr/mapasculturais}} - 
 um software livre que permite o mapeamento cultural e aprimoramento da gestão 
 cultural dos municípios e estados.
 \item \textbf{SIMEC (Sistema Integrado de Monitoramento Execução e Controle)}
 \footnote{\url{https://github.com/culturagovbr/docker-simec}} - é um portal 
 operacional e de gestão do inicialmente desenvolvida pelo MEC, que trata do 
 orçamento e monitoramento das propostas on-line do governo federal usada pelo 
 Ministério da Cultura.
 \item \textbf{SisTel (Sistema de gestão de serviços de telefonia)}
 \footnote{\url{https://github.com/culturagovbr/SisTel}} - sistema interno de 
 gestão de serviços de telefonia.
 \item \textbf{GOC (Gestão de Ouvidoria Governamental)}
 \footnote{\url{https://github.com/culturagovbr/GOG}} - sistema de ouvidoria do 
 Ministério da Cultura.
 \item \textbf{Tainacan - Sistema de Gestão de Acervos Digitais}\footnote{\url{http://tainacan.org/}} - trata-se de uma plataforma que permeia o Projeto de Política Nacional de Acervos Digitais,
 e é composto
 por módulos que podem auxiliar no gerenciamento de repositórios, ontologias, documentos e museus.
\end{itemize}

Os softwares livres desenvolvidos e/ou mantidos pelo Ministério da Cultura, em conjunto com comunidades de
usuários e cidadãos interessados, são disponibilizados na organização na plataforma github\footnote{\url{https://github.com/culturagovbr}}.
Os repositórios presentes na organização MinC não possuem uma padronização: muitos deles tem pouca ou nenhuma documentação, alguns nem possuem licenças
de software, testes automatizados, integração contínua, metricas de qualidade de código. A pouca conformidade com os modelos seguidos por
comunidades de software livre, dificulta ou limita a contribuição de interessados em coloborar com os sistemas MinC.

O aumento exponencial na capacidade de processamento e armazenamento nos
últimos anos permitiu com que um enorme volume de dados sobre a presença e interação
dos usuários na internet sejam armazenadas e catalogadas. Trata-se de um constante
fluxo de informação, usualmente pouco estruturada e distribuída por um grande número
de usuários e serviços. Elementos que individualmente fornecem pouca informação
podem ser tratados coletivamente para traçar perfis bastante precisos de grupos e usuários [Kosinski, et al, 2015].
Algoritmos e técnicas de aprendizado de máquina são aplicados para identificar padrões,
classificar, agrupar e reduzir a dimensionalidade de dados, além desenvolver sistemas de recomendações.
O amadurecimento desses algoritmos, e a disponibilização de bibliotecas, APIs e ferramentas de aprendizado de máquina 
com software livre tem possibilitado o uso tanto em projetos de pesquisa quanto no desenvolvimento de produtos de software.

Neste contexto, a colaboração com o LAPPIS configura um arranjo de pesquisa, de ação e inovação sobre a realidade da engenharia
de software e comunidades de software livre em organizações da Administração Pública Federal. Este projeto visa pesquisar
e aplicar técnicas, metodologias de desenvolvimento de software, além de aferição qualidade produto de software, em ambiente experimental 
para o desenvolvimento de software livre. 

\subsection{Objetivo Gerais}

A presente colaboração objetiva realizar pesquisas aplicadas ao ambiente e contexto de software do Ministério da Cultura de forma a investigar uma estrutura computacional que proveja suporte a análise de dados de sistemas de softwares culturais com vistas à apoiar a análise e tomada de decisões técnico, gerenciais e de negócio do Ministério.
Para tanto, espera-se estruturar na Universidade de Brasília um núcleo de pesquisa em tecnologias inovadoras voltadas à atividades culturais que pesquise e desenvolva tais tecnologias, metodologias de forma a prover suporte teórico e técnico e/ou gerencial para aplicação em estudos de observação de projetos do MinC. 

Além disso, serão aplicadas metodologias de desenvolvimento de software, além de aferição qualidade produto de software, a partir da observação participativa do pesquisador, em ambiente real da indústria, além do ambiente experimental e de desenvolvimento do Laboratório Avançado de Pesquisa, Produção e Inovação em Software (LAPPIS).

Tais pesquisas e práticas fornecerão subsídios para pertinente aproximação entre a Universidade de Brasília e o Ministério da Cultura, de forma que, os resultados das pesquisas sejam usadas para subsidiar este Ministério de ferramentas de gestão e desenvolvimento
de software colaborativo, aberto e contínuo, baseado em evidências, em diferentes arranjos produtivos, aprimorando os mecanismos de governança digital das tecnologias do portifólio;
além de fornecer software que apoiem a execução da lei Lei 8.313/91,
conhecida como Rouanet e das demais políticas de fomento e incentivo à cultura. 

\subsection{Objetivos Específicos}

Para alcançar esses objetivos gerais, que tornam esse projeto desafiador, alguns objetivos específicos são elencados como forma de decompor a complexidade do projeto. Pesquisas e desenvolvimento nas seguintes áreas serão realizados: 

\begin{itemize}
\item Realizar estudos de algoritmos de aprendizado de máquina para analisar dados da execução da Lei Rouanet;
\item Realizar estudos de métodos/práticas ágeis e de desenvolvimento lean de software, além das práticas de engenharia de software e de governança utilizadas nas comunidades de software livre, de forma a prover uma infraestrutura computacional para desenvolvimento e experimentação contínua de software;
\item Fornecer suporte tecnológico para apropriação das informações por parte da sociedade civil de maneira a contribuir para transparência pública e participação social;
\item Fornecer suporte tecnológico para estimular a participação da sociedade civil na governança digital em torno das tecnologias livres do portfólio do ministério;
\item Mineração em repositórios de software para extração e análise de dados;
\item Processamento de linguagem natural dos dados extraídos dos diferentes sistemas de software culturais;
\item Transferência de conhecimento da academia para o Estado;
\item Formação de alunos de graduação em pós-graduação em projetos com problemas reais do contexto cultural;
\item Contribuir para o fomento da cultura de software livre na Administração Pública Federal;
\item Contribuir para o desenvolvimento da cultura de tomada de decisões orientadas a dados e evidência;
\item Contribuir para o estabelecimento da cultura de desenvolvimento e experimentação contínua.
\end{itemize}

\subsection{Lei da inovação}

O presente termo de colaboração se baseia nos termos da LEI No 10.973, DE 2 DE DEZEMBRO DE 2004 (Lei da Inovação) que regula a interação 
entre nas Instituições Científica, Tecnológica e de Inovação (ICTs) e empresas ou instituições públicas ou de direito privado. A lei se
baliza no princípio de estimular o papel dos ICTs em fomentar a capacitação tecnológicas, autonomia e o desenvolvimento de produtos e processos 
inovadores no ambiente produtivo nacional. 

O presente projeto pressupõe a criação de um conjunto de softwares que serão utilizados pelo Ministério da Cultura como estratégia 
de participação social e interação democrática com a sociedade. Ainda que a lei faça previsões sobre a tutela de patentes e
propriedade intelectual, o presente acordo prevê a utilização de tecnologias e licenças de software livre que modificam um pouco a lógica usual
da atribuição da propriedade intelectual. O MinC vê como principal produto deste termo de descentralização, não o software em si, mas sim o processo
de desenvolvimento de software possibilitado pelas ferramentas desenvolvidas.
Desta forma, não existe interesse por parte do MinC em deter a propriedade intelectual das 
ferramentas com vias de comercializá-la para terceiros. Seguindo esta lógica, todo o software desenvolvido será disponibilizado sob licenças livres
com a propriedade intelectual atribuída de forma difusa entre os desenvolvedores e a Universidade de Brasília. Este arranjo permite a utilização 
da plataforma e seus sub-produtos tecnológicos por qualquer ente público ou privado interessado, inclusive com a possibilidade de comercialização
de produtos baseados nestas tecnologias. Tais tecnologias também oferecem vantagens econômicas no sentido de não dependerem de tecnologias
proprietárias que requerem o pagamento contínuo de licenças de uso.

Do ponto de vista da Lei da Inovação, as principais contrapartidas esperadas pelo ministério são:
\begin{itemize}
 \item Inovação no processo de análise dos dados dos projetos fomentados via Lei Rouanet por meio da plataforma SALIC com auxilio 
 de ferramentas de aprendizado de máquina;
 \item Estratégia/modelo de transformação de softwares legados do portifólio MinC em comunidades de software aberto;
 \item \textit{Catálogo de Software} - plataforma para disponibilizar as diversas soluções informatizadas utilizadas pelo Ministério da
 Cultura. O catálogo será desenvolvido segundo as práticas mais modernas de engenharia de software e é uma ferramenta que promove visibilidade
 do portfólio de produtos de softwares desenvolvidos por uma instituição. Será desenvolvido como software livre, 
 e pode ser utilizado para outras instituições, inclusive pela Universidade de Brasília.
 \item criação de vários produtos de software que formarão a plataforma de acompanhamento automático no SALIC; 
 \item capacitação e transferência de capital intelectual e tecnológico para o ministério com vias a permitir autonomia tecnológica 
 do MinC no uso e manutenção da plataforma e tecnologias de software associadas; 
 \end{itemize}

Aqui entende-se inovação, criação, capital intelectual e extensão tecnológica respectivamente nos termos dos art. 2o , 
incisos IV, II, XIV, XIII da referida lei.

Os produtos desenvolvidos neste termo de descentralização atendem tanto ao interesse estratégico do ministério, quanto a uma série de interesses 
em pesquisa, ensino e extensão do LAPPIS. A Universidade de Brasília, através do LAPPIS, se beneficia não só pelo aporte de capital
na forma de bolsas e serviços essenciais para a execução do projeto, mas principalmente por fortalecer a linha de atuação do laboratório 
em tecnologias de desenvolvimento de software livre, e aprendizado de máquina como explicitado no resto 
deste documento.

É importante ressaltar que o LAPPIS adota uma abordagem de Pesquisa-Ação em engenharia de software onde é necessário partir de
soluções concretas de produtos software para que seja possível estudar os processos de desenvolvimento e os produtos de software obtidos. 
Configura-se uma abordagem necessariamente mista entre pesquisa e inovação. Desta forma, busca-se atender tanto às diretrizes para projetos 
de pesquisa na Universidade de Brasília quanto às diretrizes para projetos de inovação estabelecidas pelo Centro de Apoio ao Desenvolvimento 
Tecnológico (CDT) que atua como o Núcleo de Inovação Tecnológica (NIT) da universidade.

Visto que a propriedade intelectual dos produtos de software desenvolvidos ficará em posse da universidade, ainda que sob licença livre, 
entendeu-se que é importante também observar as diretrizes do NIT da universidade na figura do CDT.
